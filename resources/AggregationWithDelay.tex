\documentclass[11pt]{article}

\usepackage{graphics,enumitem,epsfig,textcomp}
\usepackage{amsfonts,amsmath,amssymb,amsthm}
\usepackage{euscript,color,mathrsfs}
\renewcommand{\topfraction}{.95}

\usepackage{url}
\usepackage{hyperref}
\hypersetup{
    colorlinks=true,
    linkcolor=blue,
    filecolor=blue,      
    urlcolor=blue,
    citecolor=blue,
    }

\usepackage[mathlines]{lineno}
%\linenumbers

%\usepackage[centering,includeheadfoot,margin=2.5 cm]{geometry}
\usepackage[margin=1in]{geometry}
%\usepackage[utf8]{inputenc}	
%\usepackage[caption=false,subrefformat=parens,labelformat=parens]{subfig}
%\usepackage[symbol]{footmisc}


\bibliographystyle{plain}
%\usepackage{biblatex}
%\addbibresource{bibmemory.bib}

%\usepackage[square,numbers]{natbib}



%     +-----------------------------------------------------------+
%     |    Sectionwise numbering of formulas, figures, theorems   |
%     +-----------------------------------------------------------+
\numberwithin{equation}{section}
\newtheorem{defin}{Definition}%[section]
\newtheorem{theorem}{Theorem}%[section]
\newtheorem{notice}{Notice}
\newtheorem{lemma}{Lemma}%[section]
\newtheorem{proposition}{Proposition}%[section]
\newtheorem{corollary}{Corollary}%[section]
\newtheorem{example}{Example}%[section]
\newtheorem{remark}{Remark}%[section]
\newtheorem{conj}{Conjecture}
\def\begproof{\noindent{\bf Proof: }}
\def\endproof{\par\rightline{\vrule height5pt width5pt depth0pt}\medskip}
%     +------------------+
%     |     Operators    |
%     +------------------+
\def\div{\nabla\cdot}
\def\diver{\div}
\def\rot{\nabla\times}
\def\sign{{\rm sign}}
\def\arsinh{{\rm arsinh}}
\def\arcosh{{\rm arcosh}}
\def\diag{{\rm diag}}
\def\const{{\rm const}}
\def\d{\,\mathrm{d}}
%     +-----------------------------------------+
%     |      Redefinition of greek letters      |
%     +-----------------------------------------+
%\def\eps{\varepsilon}
%\def\phi{\varphi}
%\def\theta{\vartheta}
%     +-------------------------------------------+
%     |      The sets C, R, Q, M, N, P and Z      |
%     +-------------------------------------------+
\def\N{\mathbb{N}}
\def\R{\mathbb{R}}
\def\C{\hbox{\rlap{\kern.24em\raise.1ex\hbox
      {\vrule height1.3ex width.9pt}}C}}
\def\P{\hbox{\rlap{I}\kern.16em P}}
\def\Q{\hbox{\rlap{\kern.24em\raise.1ex\hbox
      {\vrule height1.3ex width.9pt}}Q}}
\def\M{\hbox{\rlap{I}\kern.16em\rlap{I}M}}
\def\Z{\hbox{\rlap{Z}\kern.20em Z}}
%    +----------------+
%    |    Equations   |
%    +----------------+
\def\({\begin{eqnarray}}
\def\){\end{eqnarray}}
\def\[{\begin{eqnarray*}}
\def\]{\end{eqnarray*}}
%   +---------------------------+
%   |    Partial derivatives    |
%   +---------------------------+
\def\part#1#2{\frac{\partial #1}{\partial #2}}
\def\partk#1#2#3{\frac{\partial^#3 #1}{\partial #2^#3}} 
\def\mat#1{{D #1\over Dt}}
\def\grad{\nabla}
%   +-----------------------+
%   |      Norms            |
%   +-----------------------+
\def\Norm#1{\left\| #1 \right\|}
%   +-----------------------+
%   |    Miscellaneous      |
%   +-----------------------+
\def\bar{\overline}
\def\lbar{\underline}

\def\tot#1#2{\frac{\d #1}{\d #2}} 
\def\totk#1#2#3{{\frac{\d^#3 #1}{\d #2^#3}}}
\def\laplace{\Delta}
\def\d{\,\mathrm{d}}
\def\N{\mathbb{N}}
\def\R{\mathbb{R}}
\def\T{\mathbb{T}}
\def\supp{\mbox{supp }}
\def\epsilon{\varepsilon}

\def\d{\mathrm{d}}


\def\comment#1{\textcolor{blue}{\bf [#1]}}

\newcommand\sgn{\text{sgn}}

\renewcommand{\thefootnote}{\fnsymbol{footnote}}

\def\dol#1{\{1,\dots,#1\}}

\usepackage{authblk}


%\def\rev#1{\textcolor{blue}{#1}}
\def\rev#1{#1}



%\title{Long-term memory inhibits and short-term memory enhances spontaneous particle aggregation}
\title{Spontaneous particle aggregation with delay}
\date{}

\pagenumbering{arabic}


\begin{document}

\maketitle

\section{The original spontaneous aggregation model}

\noindent
The individual-based stochastic model introduced in reference~\cite{BHW:2012:PhysD}
under the name `direct aggregation model' consists of a group of $N \ge 2$ biological agents (cells or animals),
characterized by their positions ${\mathbf x}_i(t)\in\R^d$, with spatial dimension $d \in \{1,2,3\}$ and $i\in [N]$,
where we have denoted the set of indices by $[N] := \{1,2,\ldots,N \}$. Every individual senses the average 
density of its close neighbours, given by
\begin{equation}   
\label{theta_i}
\vartheta_i(t) = \frac{1}{N-1} \sum_{j\neq i} W({\mathbf x}_i(t)-\mathbf{x}_j(t)) \,,\qquad \mbox{for} \quad i \in [N]\,,
\end{equation}
where $W({\mathbf x}) = w(|{\mathbf x}|)$ with the weight function $w: \R^+ \to \R^+$ assumed to be bounded, nonnegative, nonincreasing and
integrable on $\R^d$. Without loss of generality we impose the normalization
\begin{equation}  
\label{eq:Wnorm}
\int_{\R^d} W({\mathbf x}) \, \d {\mathbf x} = 1 \,. 
\end{equation}
A generic example of $w$ is the (properly normalized) characteristic function
of the interval $[0,R]$, corresponding to the sampling radius $R > 0$.
The average density $\vartheta_i$ is then simply the fraction
of individuals located within the distance $R$ from the $i$-th individual.
The individual positions are subject to a random walk with modulated amplitude,
described by the system of coupled SDEs
\begin{equation}
\label{model1}
\d {\mathbf x}_i(t) \,=\, G(\vartheta_i) \, \d {\mathbf B}_i^t \,,\qquad \mbox{for} \quad i \in [N]\,,
\end{equation}
where ${\mathbf B}_i^t$ are independent $d$-dimensional Brownian motions.
The \emph{response function} $G:\R^+ \to\R^+$ is assumed to be globally bounded, nonnegative and decreasing.
The monotonicity of $G$ is implied by the modeling assumption that the individuals respond to 
higher perceived population densities in their vicinity by reducing the amplitude of 
their random walk.


\section{Spontaneous aggregation model with delay}
\noindent

From the modeling point of view it makes sense to consider two types of delay:

\begin{itemize}
\item
\textbf{Transmission-type delay}, where we assume that there is non-negligible time
for the transmission of information from agent $x_j$ to agent $x_i$.
I.e., agent $i$ at time $t$ makes its decision based on the information about agent $j$
that was actual at time $t-\tau$, where $\tau>0$ is the transmission delay.
For simplicity we shall assume that $\tau>0$ is a global constant, taking the same value
for all pairs of agents $i$, $j$.
This means that formula \eqref{theta_i} is replaced by
\begin{equation} 
\label{theta_i_trans}
   \vartheta_i(t) = \frac{1}{N-1} \sum_{j\neq i} W({\mathbf x}_i(t)-\mathbf{x}_j(t-\tau)) \,,\qquad \mbox{for} \quad i \in [N]\,.
\end{equation}

\item
\textbf{Reaction-type delay}, where we assume that the information transmission is instantaneous
(or, strictly speaking, the transmission delay is negligible),
but the agents need a non-negligible time to carry out their actions.
This means that agent $i$ at time $t$ reacts to the information $x_i-x_j$ that
was actual at time $t-\tau$.
In this case we replace formula \eqref{theta_i} by
\begin{equation} 
\label{theta_i_react}
   \vartheta_i(t) = \frac{1}{N-1} \sum_{j\neq i} W({\mathbf x}_i(t-\tau)-\mathbf{x}_j(t-\tau)) \,,\qquad \mbox{for} \quad i \in [N]\,.
\end{equation}
\end{itemize}


\bibliography{bibmemory}
%\printbibliography 

\end{document}
